\documentclass{sigish}
\begin{document}

% ------ Enter task number (1-3) here
\def\taskno{1}

% ------ Enter your group letter here
\def\groupno{X}

% ------ Enter your group size here
\numberofauthors{5}

% ------ Enter details for each group member here
\author{
% 1st. group member
\alignauthor First Author\\
       \affaddr{Mat.nr.: 0000000}\\
       \email{k0000000@student.jku.at}\\ % or whichever you prefer
       \affaddr{Score Percentage: 20\%}\\ % adapt accordingly
% 2nd. group member
\alignauthor Second Author\\
       \affaddr{Mat.nr.: 0000000}\\
       \email{k0000000@student.jku.at}\\
       \affaddr{Score Percentage: 20\%}\\
\and
% 3rd. group member
\alignauthor Third Author\\
       \affaddr{Mat.nr.: 0000000}\\
       \email{k0000000@student.jku.at}\\
       \affaddr{Score Percentage: 20\%}\\
% 4th. group member
\alignauthor Fourth Author\\
       \affaddr{Mat.nr.: 0000000}\\
       \email{k0000000@student.jku.at}\\
       \affaddr{Score Percentage: 20\%}\\
% 5th. group member   %   delete if only 4 group members
\alignauthor Fifth Author\\
       \affaddr{Mat.nr.: 0000000}\\
       \email{k0000000@student.jku.at}\\
       \affaddr{Score Percentage: 20\%}\\
}
% leave untouched!
\title{Learning from User-generated Data SS2016 -- Task \taskno}
\subtitle{Group \groupno}
\maketitle


% The following is just an example structure. Change to whatever works best for you. 
% However, make sure to include everything requested in the exercise description.

\section{Description of Task \taskno}

Task to be performed.

\section{Data Analysis}

What's in the data?

%\begin{figure}
%\centering
%\includegraphics[width=\columnwidth]{graph.eps}
%\caption{Distribution of ratings.}
%\vskip -6pt
%\end{figure}

\section{Proposed Method}

How did you address this task?
Description of used algorithms.

\section{Experimental Setup}

How can you measure which algorithm was best? In comparison to what? How do you make sure that you do not use cases for testing that you have used for learning?


\section{Results}

RMSE values of different approaches and parameters.


\section{Conclusions}

Which run did you submit? Why?
Further interesting encounters.


% The following two commands are all you need in the
% initial runs of your .tex file to
% produce the bibliography for the citations in your paper.
\bibliographystyle{abbrv}
\bibliography{refs}  % refs.bib is the name of the Bibliography in this case
% You must have a proper ".bib" file
%  and remember to run:
% latex bibtex latex latex
% to resolve all references

\end{document}
